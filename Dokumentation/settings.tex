\documentclass[fontsize=12pt, paper=a4, headinclude, twoside=false, 
parskip=half+,
pagesize=auto, numbers=noenddot, open=right, toc=listof, toc=bibliography]{scrreprt}

\usepackage{etex}% bei vielen packages


% PDF-Kompression
\pdfminorversion=5
\pdfobjcompresslevel=1

% CSV
%\usepackage{pgfplotstable}

% Allgemeines
\usepackage[automark]{scrpage2} % Kopf- und Fußzeilen
\usepackage{amsmath,marvosym} % Mathesachen
\usepackage[T1]{fontenc} % Ligaturen, richtige Umlaute im PDF
\usepackage[utf8]{inputenc}% UTF8-Kodierung für Umlaute usw
%\usepackage[ansinew]{inputenc}%windows
\usepackage{blindtext}


% Schriften
\usepackage{mathpazo} % Palatino für Mathemodus
%\usepackage{mathpazo,tgpagella} % auch sehr schöne Schriften
\usepackage{setspace} % Zeilenabstand
\onehalfspacing % 1,5 Zeilen


% Schriften-Größen
\setkomafont{chapter}{\Huge\rmfamily} % Überschrift der Ebene
\setkomafont{section}{\Large\rmfamily}
\setkomafont{subsection}{\large\rmfamily}
\setkomafont{subsubsection}{\large\rmfamily}
\setkomafont{chapterentry}{\large\rmfamily} % Überschrift der Ebene in Inhaltsverzeichnis
\setkomafont{descriptionlabel}{\bfseries\rmfamily} % für description Umgebungen




%\usepackage{dirtree}
% Sprache: Deutsch
\usepackage[ngerman]{babel} % Silbentrennung


% PDF
 \usepackage[ngerman,  pdfauthor={David Drost, Edward Kupfer, Hendrik Sawade, Tobias Wenzel}, pdftitle={Digital Text Forensics Search}, breaklinks=true]{hyperref}


\usepackage[final,stretch=40]{microtype} % mikrotypographische Optimierungen
\usepackage{url} % ermögliche Links (URLs)
\usepackage{pdflscape} % einzelne Seiten drehen können


% Tabellen

\usepackage{tabularx} % Für Tabellen mit vorgegeben Größen
\usepackage{longtable} % Tabellen über mehrere Seiten
\usepackage{array}%% 


%%  Bibliographie


\usepackage[round,authoryear]{natbib}
\bibliographystyle{mlu_ifg}
%\usepackage{bibgerm} % Umlaute in BibTeX
%\addbibresource{bibliography.bib}


\usepackage{float}

% Bilder
\usepackage{graphicx} % Bilder
\usepackage{color} % Farben
\graphicspath{{./img/}}
\DeclareGraphicsExtensions{.pdf,.png,.jpg} % bevorzuge pdf-Dateien
\usepackage{subfig}% http://ctan.org/pkg/subfig % Das scheint das neue zu sein.

\usepackage{caption}

\captionsetup{font=small,labelfont=bf,format=plain,labelsep=colon,justification=justified}
\captionsetup[subtable]{position=top}



% Quellcode
\usepackage{listings} % für Formatierung in Quelltexten
\definecolor{grau}{gray}{0.25}
\lstset{
	extendedchars=true,
	basicstyle=\tiny\ttfamily,
	%basicstyle=\footnotesize\ttfamily,
	tabsize=2,
	keywordstyle=\textbf,
	commentstyle=\color{grau},
	stringstyle=\textit,
	numbers=left,
	numberstyle=\tiny,
	% für schönen Zeilenumbruch
	breakautoindent  = true,
	breakindent      = 2em,
	breaklines       = true,
	postbreak        = ,
        showstringspaces=false,
	prebreak         = \raisebox{-.8ex}[0ex][0ex]{\Righttorque},
}
% Pseudocode
\usepackage{algorithm}
\usepackage[]{algpseudocode}
% linksbündige Fußboten
\deffootnote{1.5em}{1em}{\makebox[1.5em][l]{\thefootnotemark}}

\typearea{14} % typearea berechnet einen sinnvollen Satzspiegel (das heißt die Seitenränder) siehe auch http://www.ctan.org/pkg/typearea. Diese Berechnung befindet sich am Schluss, damit die Einstellungen oben berücksichtigt werden

\usepackage{scrhack} % Vermeidung einer Warnung


% Eigene Befehle %%%%%%%%%%%%%%%%%%%%%%%%%%%%%%%%%%%%%%%%%%%%%%%%%5



\usepackage{import} %zum importieren von Grafiken mit dem richtigen Pfad
\usepackage[%
%disable,  % uncomment this line to hide the comments!
textsize=footnotesize
]{todonotes}
\usepackage{xspace}
\newcommand{\comment}[1]{\todo[color=blue!40]{#1}\xspace{}}
\usepackage{soul}


% Abkürzungsverzeichnis
\usepackage{acronym}


% Sonstiges
\usepackage{booktabs}
\usepackage{arydshln}
\usepackage{eurosym}

\newcommand{\rowgroup}[1]{\hspace{-1em}#1}



\newlength{\spaltenbreite}
\spaltenbreite6cm
\usepackage{array}
\usepackage{booktabs,colortbl}
\definecolor{Gray}{gray}{0.9}
\usepackage{siunitx} % Formats the units and values
\usepackage{rotating}
\usepackage{physics}
\DeclareMathOperator*{\argmin}{arg\,min}
\DeclareMathOperator*{\argmax}{arg\,max}


\lstdefinestyle{mystyle}{
  basicstyle=%
    \ttfamily
    \lst@ifdisplaystyle\normalsize\fi
}
\lstset{style=mystyle}


% Weitere spezifische Pakete und Befehle
\usepackage{pgfplots}
\newcommand\gauss[2]{1/(#2*sqrt(2*pi))*exp(-((x-#1)^2)/(2*#2^2))} 
%penalty100 macht, dass Matlab in der nächsten Zeile landet.

% Showframe zeigt dir den Rand an. Dann kannt du sehen, ob irgendwas übersteht
%\usepackage{showframe}
%% Beispiele für korrekte Kodierungen.

\newcommand{\matlab}{MATLAB\textsuperscript{\textregistered}\xspace}
\newcommand{\coder}{\penalty100 MATLAB Coder\textsuperscript{\texttrademark}\xspace}
\newcommand{\prtools}{\penalty100 PRTools\xspace}
\newcommand{\fann}{\penalty100 FANN\xspace}



%ceil 
\usepackage{mathtools}
\DeclarePairedDelimiter{\ceil}{\lceil}{\rceil}

% UML
% \usepackage{tikz}
% \usepackage{../../tikz-uml}
% \usepackage{ifthen}
% \usepackage{xstring}
% \usepackage{calc}
% \usepackage{pgfopts}
% \tikzumlset{font=\tiny}


\usepackage{bytefield}
\setlength{\parfillskip}{0.5em plus 1fil} % don't fill the last line
\setlength{\emergencystretch}{.1\textwidth} % not to get preposterously bad lines

%%% Local Variables:
%%% mode: latex
%%% TeX-master: "arbeit"
%%% End:
