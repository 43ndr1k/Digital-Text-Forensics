\documentclass[fontsize=12pt, paper=a4, headinclude, twoside=false, 
parskip=half+,
pagesize=auto, numbers=noenddot, open=right, toc=listof, toc=bibliography]{scrreprt}

\usepackage{etex}% bei vielen packages


% PDF-Kompression
\pdfminorversion=5
\pdfobjcompresslevel=1

% CSV
%\usepackage{pgfplotstable}

% Allgemeines
\usepackage[automark]{scrpage2} % Kopf- und Fußzeilen
\usepackage{amsmath,marvosym} % Mathesachen
\usepackage[T1]{fontenc} % Ligaturen, richtige Umlaute im PDF
\usepackage[utf8]{inputenc}% UTF8-Kodierung für Umlaute usw
%\usepackage[ansinew]{inputenc}%windows
\usepackage{blindtext}


% Schriften
\usepackage{mathpazo} % Palatino für Mathemodus
%\usepackage{mathpazo,tgpagella} % auch sehr schöne Schriften
\usepackage{setspace} % Zeilenabstand
\onehalfspacing % 1,5 Zeilen


% Schriften-Größen
\setkomafont{chapter}{\Huge\rmfamily} % Überschrift der Ebene
\setkomafont{section}{\Large\rmfamily}
\setkomafont{subsection}{\large\rmfamily}
\setkomafont{subsubsection}{\large\rmfamily}
\setkomafont{chapterentry}{\large\rmfamily} % Überschrift der Ebene in Inhaltsverzeichnis
\setkomafont{descriptionlabel}{\bfseries\rmfamily} % für description Umgebungen




%\usepackage{dirtree}
% Sprache: Deutsch
\usepackage[ngerman]{babel} % Silbentrennung


% PDF
 \usepackage[ngerman,  pdfauthor={David Drost, Edward Kupfer, Hendrik Sawade, Tobias Wenzel}, pdftitle={Digital Text Forensics Search}, breaklinks=true]{hyperref}


\usepackage[final,stretch=40]{microtype} % mikrotypographische Optimierungen
\usepackage{url} % ermögliche Links (URLs)
\usepackage{pdflscape} % einzelne Seiten drehen können


% Tabellen

\usepackage{tabularx} % Für Tabellen mit vorgegeben Größen
\usepackage{longtable} % Tabellen über mehrere Seiten
\usepackage{array}%% 


%%  Bibliographie


\usepackage[round,authoryear]{natbib}
\bibliographystyle{mlu_ifg}
%\usepackage{bibgerm} % Umlaute in BibTeX
%\addbibresource{bibliography.bib}


\usepackage{float}

% Bilder
\usepackage{graphicx} % Bilder
\usepackage{color} % Farben
\graphicspath{{./img/}}
\DeclareGraphicsExtensions{.pdf,.png,.jpg} % bevorzuge pdf-Dateien
\usepackage{subfig}% http://ctan.org/pkg/subfig % Das scheint das neue zu sein.

\usepackage{caption}

\captionsetup{font=small,labelfont=bf,format=plain,labelsep=colon,justification=justified}
\captionsetup[subtable]{position=top}



% Quellcode
\usepackage{listings} % für Formatierung in Quelltexten
\definecolor{grau}{gray}{0.25}
\lstset{
	extendedchars=true,
	basicstyle=\tiny\ttfamily,
	%basicstyle=\footnotesize\ttfamily,
	tabsize=2,
	keywordstyle=\textbf,
	commentstyle=\color{grau},
	stringstyle=\textit,
	numbers=left,
	numberstyle=\tiny,
	% für schönen Zeilenumbruch
	breakautoindent  = true,
	breakindent      = 2em,
	breaklines       = true,
	postbreak        = ,
        showstringspaces=false,
	prebreak         = \raisebox{-.8ex}[0ex][0ex]{\Righttorque},
}
% Pseudocode
\usepackage{algorithm}
\usepackage[]{algpseudocode}
% linksbündige Fußboten
\deffootnote{1.5em}{1em}{\makebox[1.5em][l]{\thefootnotemark}}

\typearea{14} % typearea berechnet einen sinnvollen Satzspiegel (das heißt die Seitenränder) siehe auch http://www.ctan.org/pkg/typearea. Diese Berechnung befindet sich am Schluss, damit die Einstellungen oben berücksichtigt werden

\usepackage{scrhack} % Vermeidung einer Warnung


% Eigene Befehle %%%%%%%%%%%%%%%%%%%%%%%%%%%%%%%%%%%%%%%%%%%%%%%%%5



\usepackage{import} %zum importieren von Grafiken mit dem richtigen Pfad
\usepackage[%
%disable,  % uncomment this line to hide the comments!
textsize=footnotesize
]{todonotes}
\usepackage{xspace}
\newcommand{\comment}[1]{\todo[color=blue!40]{#1}\xspace{}}
\usepackage{soul}


% Abkürzungsverzeichnis
\usepackage{acronym}


% Sonstiges
\usepackage{booktabs}
\usepackage{arydshln}
\usepackage{eurosym}

\newcommand{\rowgroup}[1]{\hspace{-1em}#1}



\newlength{\spaltenbreite}
\spaltenbreite6cm
\usepackage{array}
\usepackage{booktabs,colortbl}
\definecolor{Gray}{gray}{0.9}
\usepackage{siunitx} % Formats the units and values
\usepackage{rotating}
\usepackage{physics}
\DeclareMathOperator*{\argmin}{arg\,min}
\DeclareMathOperator*{\argmax}{arg\,max}


\lstdefinestyle{mystyle}{
  basicstyle=%
    \ttfamily
    \lst@ifdisplaystyle\normalsize\fi
}
\lstset{style=mystyle}


% Weitere spezifische Pakete und Befehle
\usepackage{pgfplots}
\newcommand\gauss[2]{1/(#2*sqrt(2*pi))*exp(-((x-#1)^2)/(2*#2^2))} 
%penalty100 macht, dass Matlab in der nächsten Zeile landet.

% Showframe zeigt dir den Rand an. Dann kannt du sehen, ob irgendwas übersteht
%\usepackage{showframe}
%% Beispiele für korrekte Kodierungen.

\newcommand{\matlab}{MATLAB\textsuperscript{\textregistered}\xspace}
\newcommand{\coder}{\penalty100 MATLAB Coder\textsuperscript{\texttrademark}\xspace}
\newcommand{\prtools}{\penalty100 PRTools\xspace}
\newcommand{\fann}{\penalty100 FANN\xspace}

\newcommand{\pdfbox}{Apache PDFBox\textsuperscript{\textregistered}\xspace}

\newcommand{\tika}{\penalty100 Apache Tika\textsuperscript{\texttrademark}\xspace}


%ceil 
\usepackage{mathtools}
\DeclarePairedDelimiter{\ceil}{\lceil}{\rceil}

% UML
% \usepackage{tikz}
% \usepackage{../../tikz-uml}
% \usepackage{ifthen}
% \usepackage{xstring}
% \usepackage{calc}
% \usepackage{pgfopts}
% \tikzumlset{font=\tiny}


\usepackage{bytefield}
\setlength{\parfillskip}{0.5em plus 1fil} % don't fill the last line
\setlength{\emergencystretch}{.1\textwidth} % not to get preposterously bad lines

%%% Local Variables:
%%% mode: latex
%%% TeX-master: "arbeit"
%%% End:
 % Importiere die Einstellungen aus der 




% hier beginnt der eigentliche Inhalt
\begin{document}
\pagenumbering{Roman} % Seitenummerierung mit großen römischen Zahlen 
\pagestyle{empty} % kein Kopf- oder Fußzeilen auf den ersten Seiten

% Titelseite
\clearscrheadings\clearscrplain
\begin{center}

  \includegraphics[width=.4\textwidth]{./img/uni_leipzig_logo}\\

  
  \begin{Large}
    \textbf{Digital Text Forensics Search}\\
  \end{Large}
  \vspace{8 mm}
 Information Retrieval\\ WS 2017/2018\\
  

	\vfill
	\normalsize
	\newcolumntype{x}[1]{>{\raggedleft\arraybackslash\hspace{0pt}}p{#1}}
	\begin{tabular}{x{6cm}p{7.5cm}}
		\rule{0mm}{5ex}\textbf{Betreurer:} & Martin Potthast
		\newline martin.potthast@uni-leipzig.de\\
          
		\rule{0mm}{5ex}\textbf{Autor:} & David Drost
		\newline dd42cequ@studserv.uni-leipzig.de
		\newline Matrikelnummer:  \\ 
		
		\rule{0mm}{5ex}\textbf{Autor:} & Edward Kupfer
		\newline ek96foje@studserv.uni-leipzig.de
		\newline Matrikelnummer:  \\ 

		\rule{0mm}{5ex}\textbf{Autor:} & Hendrik Sawade  
		\newline hs34byhe@studserv.uni-leipzig.de 
		\newline Matrikelnummer: 3745956 \\ 
                
		\rule{0mm}{5ex}\textbf{Autor:} & Tobias Wenzel 
		\newline tw54byka@studserv.uni-leipzig.de 
		\newline Matrikelnummer: 3733301 \\ 
	
		
		\rule{0mm}{5ex}\textbf{Abgabedatum:} & Leipzig, den \today \\ 
	\end{tabular} 

\end{center}
\clearpage

%%% Local Variables:
%%% mode: latex
%%% TeX-master: "arbeit"
%%% End:


\pagestyle{useheadings} % normale Kopf- und Fußzeilen für den Rest
\setcounter{tocdepth}{3}    % 3 = schließe auch subsubsections ein
\setcounter{secnumdepth}{3} % 3 = verleihe auch Bezifferung für subsubsections

\tableofcontents % erstelle hier das Inhaltsverzeichnis



%%% könnt ihr gerne benutzen, seh hier aber keinen Mehrwert.

%% \addchap{Abkürzungsverzeichnis}
%% \begin{acronym}
%%   %%\acro{NN}{Neural Network, dt. Neuronales Netzwerk}
%% \end{acronym}



\pagenumbering{arabic} % ab jetzt die normale arabische Nummerierung


\chapter{Einleitung}\label{ch:intro}
Ziel des Praktikums war es, eine themenspezifische Suchmaschine zu implementieren, die Wissenschaftliche Publikationen aus dem Fachgebiet der Digital Text Forensics indexieren und durchsuchen kann. 
In der Digital Text Forensics werden Methoden und Verfahren zur Identifikation von Authoren, Erkennung von Plagiaten und Profilerstellung von Authoren untersucht und bereitgestellt. 
Die Publikationen lagen hauptsächlich im PDF-Format und in englischer Sprache vor, weshalb Dokumente in anderen Formaten und Sprachen nicht berücksichtigt wurden. 
Im Verlauf der Vorverarbeitung wurden die Dokumente in ein einheitliches Format und eine einheitliche Zeichenkodierung überführt. 
Um einzelne Bestandteile bzw. Felder eines Dokuments, wie z.B. Titel oder Fließtext, getrennt betrachten zu können, wurde XML als Format gewählt, in das alle Dokument konvertiert und zur Indexierung genutzt werden. 
Die Erkennung und Extraktion einzelner Bestandteile eines Dokuments war dabei eine wesentliche Aufgabe, da Titel der Publikationen als Suchresultat angezeigt werden sollen, diese aber kaum als Meta-Informationen in den PDF-Dokumenten vorhanden waren. 
Das Vorkommen von Wörtern der Suchanfrage in bestimmten Feldern kann somit beim Ranking berücksichtigt werden: Wenn Wörter der Suchanfrage im Titel eines Dokuments vorkommen, wird es als relevanter bewertet, als wenn sie im Fließtext vorkommen. 
Weitere Parameter, die in die Gewichtung der Relevanz eines Dokuments eingehen sind Logdaten, z.B. wie oft ein Dokument für eine Suchanfrage geklickt wurde, sowie die Zeitdauer, die Nutzer auf einem Dokument verbracht haben.
Die Aufzeichnung dieser Daten wurden in einer Datenbank im Backend implementiert, welche auch genutzt wurde, um die Autovervollständigung von Suchanfragen zu ermöglichen. 
Weiterhin geht in die Gewichtung eines Dokuments die Anzahl der Erwähnungen des Dokuments in anderen Publikationen mit ein. 
Zur Ermittlung dieses Kennwertes wurde ein Pearl-Skript geschrieben, sodass dieser Wert bei Erweiterung des Indexes durch neue Dokumente, aktualisiert werden kann. 
Oben genannte Komponenten der Suchmaschine wurden unter Verwendung verschiedener Programmbilbiotheken implementiert, darunter Apache PDFBox, Tika und Docear bei der Vorverarbeitung, Apache Lucene für Indexierung und Suche und das Spring Framework für Backend und Frontend. 
Das Frontend dient als Interface für Suchanfragen eines Nutzers und zur Präsentation der Suchergebnisse, geordnet nach Relevanz und mit einem kurzen Textausschnitt (Snippet) für jedes Suchergebnis. 
In den folgenden Kapiteln wird auf die einzelnen Komponenten der Suchmaschine detailliert eingegangen. 

%% Hier schreibt man dann was. Und so \comment{kommentiert ihr Sachen.}. Offenbar sieht das nicht so schön aus.


%% \begin{figure}[ht]
%%   \subfloat[Pyramidenzelle \label{fig:pyramid_cell}]{%
%%     \includegraphics[width=0.4\linewidth]{aktionspotential}
%%   }
%%   \hfill
%%   \subfloat[Neocortex \label{fig:neocortex}]{%
%%     \includegraphics[width=0.5\textwidth]{leitungsgeschwindigkeit}
%%   }
%%   \caption[Aktionspotential und Leitungsgeschwindikeit]{Links:
%%     AP. Rechts: Leitungsgeschwindikeit von verschiedenen Fasern.}
%% \label{fig:ap_leitung}
%% \end{figure}



%%% Local Variables:
%%% mode: latex
%%% TeX-master: "../arbeit"
%%% End:


\chapter{Vorverarbeitung}\label{ch:prepros}



Im folgenden Abschnitt wird beschrieben, wie der betrachtete Datensatz in eine indizierbare Form
gebracht wird. Zunächst wird in \autoref{sec:pdfextraction} ein
Überblick über die Auswahl der Fremdsoftware der Vorverarbeitungstools
gegeben und auf die einzelnen Schritte der Verarbeitung eingegangen. Anschließend wird
in \autoref{sec:ranking} ein Verfahren zur Berechnung eines Rankings vorgestellt.

% \section{Komponenten}\label{sec:components}


\section{Daten-Extraktion}\label{sec:pdfextraction}


\subsection{PDF-Extraktion}\label{sec:preprosselection}

Es existieren einige Toolboxen, die die Extraktion von Text und
Meta-Informationen vornehmen können.  Aufgrund der Wahl der
Programmiersprache Java wurde dazu das \tika Toolkit in Betracht
gezogen. Es bietet einfach gestaltete Schnittstellen zu Open Source
Libraries um je nach Datenformat eine geeignete Datenextraktion
vorzunehmen. Neben PDFs lassen sich auch DOCs, Power Point
Präsentationen und Bilder mittels OCR ansprechen. Die Komponente der
PDF-Extraktion, \pdfbox, lässt sich auch außerhalb von Tika nutzen.
Die Paper-Auswahl besteht überwiegend aus PDFs\footnote{Zusammensetzung des Datensatzes: 1595 PDFs, 11
DOCs und 1 HTML.}, deswegen wurde \pdfbox ausgewählt, um einen-Overhead
zu vermeiden. Bei der Zunahme von weiteren Formaten lässt sich der
Code leicht mit entsprechenden \tika - Methoden austauschen.

Bei der Untersuchung des Datensatzes wird deutlich, dass nur ein
geringer Anteil an Papern korrekte Meta-Informationen angegeben werden. Um
Titel aus den Texten zu extrahieren wurde deswegen Docear
PdfDataExtractor\footnote{Link:
  \url{https://www.docear.org/tag/pdf-title-extraction/}}
eingesetzt. Die Software sucht, neben weitern Heuristiken, auf der
ersten Seite des Artikels nach langen zusammenhängenden Wortsequenzen.
Aufgrund der geringen Anzahl an Papern werden die Artikel-Daten
einzeln als XML-Files gespeichert, um die Überprüfung während der
Entwicklungsphase zu erleichterten. Das Haupt-Element article enthält
die Elemente metaData mit title, authors, publificationDate, refCount
und textElements mit abstract und fullText.  Um die von der ASCII
Codierung abweichende Zeichen korrekt darzustellen, werden die
Text-Elemente als nicht interpretierte Zeichen gespeichert.
Zusätzlichen werden fileName, filePath sowie parseTime festgehalten,
die in der weiteren Verarbeitung benötigt werden.

Der dem PDF entnommene Titel wird zunächst auf zulässige Länge und
Zeichen geprüft. Verstößt die Zeichenkette gegen die Restriktionen,
wird mit Docear PdfDataExtractor versucht, im Text einen validen Titel
zu finden. Valide Titel werden an eine Schnittstelle der Digital
Bibliography \& Library Project (DBLP)\footnote{Link zu DBLP:
\url{http://dblp.uni-trier.de/}} gesendet und mit der dort
vorliegenden Datenbank verglichen. Die Attribute des Artikels mit der
höchsten Übereinstimmung für den aktuellen Artikel werden
übernommen. Alternativ lässt sich der Datensatz als XML-Datei lokal zu
durchsuchen und mit weiteren Daten anzureichern.  In
\autoref{sec:heuristic-title-search} soll weiter darauf eingegangen
werden.  Konnte nach wie vor kein valider Titel entnommen werden, wird
eine Zeichenkette festgelegter Länge als Heuristik für den Title
angenommen. In diesem Fall wird eine Named Entity Recognition auf den
ersten Wörtern des Artikel-Textes gefahren, um mögliche Autoren zu
finden. Hierzu wird Apache OpenNLP verwendet\footnote{Link zu OpenNLP:
\url{https://opennlp.apache.org/}}.



\subsection{Heuristische Titel Suche}\label{sec:heuristic-title-search}

Die extrahierten Daten zeigen auch nach den in
\autoref{sec:pdfextraction} beschriebenen Schritten ein nicht
zufriedenstellendes Ergebnis. Im folgenden soll eine heuristische
Titel Suche vorgestellt werden, die von Martin Potthast angeregt
wurde. Sie vergleicht den Text mit Attributen einer vorliegenden
Meta-Daten Kollektion. Der Algorithmus folgt der Annahme, dass die
Meta-Daten mit der höchsten Übereinstimmung die korrekten sein müssen,
wenn diese als Vergleichsdaten vorliegen. Als Daten-Quelle wurde zum
einen eine Auswahl an Papern des DBLP, die in relevanten Journalen
erscheinen und eine bereits vorhandene Auswertung von
Zitations-Analysen\footnote{Prof. Efstathio Stamatatos, Universität
der Ägäis.} zusammengefasst. Der Algorithmus durchläuft je extrahiertem
Artikel die folgende Prozedur:

Es werden $300$ Wörtern des Artikels als zu
vergleichender Text gespeichert. Während der Stax-Parser über die
Meta-Daten-Kollektion läuft, werden die Elemente als aktuelles
Artikel-Objekt gespeichert. Es wird ein Score berechnet, der eine
mögliche Übereinstimmung anzeigen soll. Enthält der entsprechende Text
Wörter der Attribute Titel, Autoren oder Publikations-Zeitpunkt,
erhält der Artikel Punkte. Genaue Übereinstimmung des Titels bzw. der
Autoren erhalten zusätzliche Boni, die relativ zur Länge der Attribute
berechnet werden. Die Bewertung korrekter Titel mit $80$, Autoren mit $20$
und korrektem Publikations-Jahr mit $5$ Punkten brachte hier annehmbare
Ergebnisse. Der Artikel mit der höchsten Punktzahl wird akzeptiert. Da
nicht garantiert werden kann, dass die korrekte Artikel Daten gefunden
werden, muss der Score einen Schwellwert von $200$ überschreiten.

Die im Algorithmus verwendeten Werte wurden empirisch ermittelt und die Ergebnisse mit
den entnommenen rohen Textteilen verglichen. Artikel
müssen mit mindestens zwei Wörter des Titels und zwei Autoren
enthalten, um als korrekt angesehen zu werden. Die gewonnenen Daten
haben dann eine hohe Wahrscheinlichkeit, korrekt zu sein. Von $1600$
Publikationen konnten $413$ erkannt werden. Die Auswertung mit
extrahierten Titel und Roh-Text liegt als XML-Datei bei (siehe \href{https://github.com/43ndr1k/Digital-Text-Forensics/blob/develop/Dokumentation/res/heuristic_search_results.xml}{Github}). Das
Verfahren kann um weitere Felder wie Journal, Verlag oder Universität
erweitert werden und bietet bei Zunahme von weiteren vor-selektierten
Meta-Daten eine sinnvolle Erweiterung zur Bestimmung korrekter Artikel-Daten.

\section{Ranking der Dokumente}\label{sec:ranking}

%\subsection{Paperrank}\label{sec:paperrank}

Die gegebenen Paper können auch zur Indizierungszeit gerankt
werden. Hierbei spielt die Relevanz zu einer Query noch keine
Rolle. Es geht darum, welches Paper generell wichtiger ist, als ein
anderes. Diese Gewichtung wird in das finale Ranking einberechnet. Es
existieren diverse Faktoren nach denen die Relevanz von Papers im
Bezug auf andere Papers festgemacht werden kann. Im Zuge der
Bearbeitung wurden zum Einen die Anzahl der Klicks und die
Verweildauer auf einem Paper einbezogen und zum Anderen die Zahl der
Zitierungen in anderen Papers. Weitere Faktoren, wie bspw. Autoren,
Sprache oder Form wurden für das Ranking nicht hinzugezogen, könnten
jedoch in weiteren Arbeiten zu dem Thema betrachtet werden.  \\

In diesem Abschnitt wird auf die Anzahl der Zitierungen für jedes
Paper eingegangen. In Anlehnung an den Begriff Pagerank wird dies hier
als Paperrank bezeichnet.  Webseiten nehmen Bezug auf andere
Webseiten. Es ist möglich daran die Wichtigkeit einer Webseite
festzumachen. Wird eine oft auf anderen Seiten erwähnt ist davon
auszugehen, dass sie wichtiger ist, als eine weniger erwähnte. Bei
Papers ist das ähnlich: Wird ein Paper oft zitiert so ist davon
auszugehen, dass es für die Autoren von Papers eine höhere Relevanz
hat, als Papers die weniger zitiert werden. Daher wurde dieser Aspekt
für das Ranking der Dokumente für die Suchmaschine
betrachtet. Außerdem kann als Nebeneffekt anhand des Paperranks die
komplette Verteilung der Zitierungen dargestellt werden.  Die
Umsetzung erfolgte über ein Perlskript, es gab Probleme bei der
Umsetzung mit Java. Das Skript kann Offline zur Indexing-Zeit
ausgeführt werden. Es wird vom Hauptprogramm über die Javaklasse
\lstinline{RunScript.class} aufgerufen. Das Skript wurde für die Perl
Version $5$ getestet\footnote{Hier bitte die Readme beachten.}. Das
Skript wurde zur besseren Nachvollziehbarkeit in einzelne Schritte
unterteilt auf die im Folgenden eingegangen wird.  Zunächst wurden zur
Feststellung einige Grunddaten aus den gegebenen wissenschaftlichen
Texten benötigt. Daraus wurden im speziellen die Titel und die Quellen
genutzt. Die Paper liegen dank der Vorverarbeitung in Form von
Textdateien vor. Die Titel sind durch die XML-Tag <title> markiert. Es
erfolgte eine Extraktion durch reguläre Ausdrücke auf den <title>-Tag
und zur Sicherheit, da nicht alle Paper einen ordentlich befüllten Tag
hatten auf den <Filename>-Tag.  Die erste Herausforderung stellte die
Extraktion der Quellen für jedes Paper dar. Diese sind nicht gesondert
markiert, sondern sind im Text eingebettet. Dies wurde mit Hilfe einer
Heuristik gelöst. Quellen sind typischer Weise nummeriert. Diese
Nummerierung wird oftmals eingeschlossen durch eckige [] oder runde ()
Klammern. Daher wurde mit Hilfe regulärer Ausdrücke nach Klammern
gesucht, welche ein- oder zweistellige Zahlen enthalten. War diese
Bedingung erfüllt, wurde die entsprechende Zeile als Quelle
extrahiert. Mit diesen Daten für Titel und Quellen wurde nun
weitergearbeitet.  Im nächsten Schritt werden die so entstandenen
Quellen in Zeichenketten zerlegt. Für eine Länge von 30 Zeichen haben
sich gute Ergebnisse eingestellt. Dies geschieht, da die Quellen nicht
nur den Titel der zitierten Quelle enthalten, sondern auch Dinge wie
bspw. Erscheinungsdatum, Author, Verlag, ISBN.  Danach können die
Zeichenketten mit den Titeln abgeglichen werden. Wenn der Titel die
Zeichenkette beinhaltet, dann wird der Titel in den nächsten
Bearbeitungsschritt übernommen. Hierbei werden beim Abgleich
Sonderzeichen und Zahlen ausgenommen.  Daraufhin
werden die Titel gezählt und sortiert. Es entsteht eine Liste mit den
Papers zusammen mit der Anzahl mit der sie in den Quellen der anderen
Paper vorkommen. Daraus kann nun entnommen werden, welche Paper die
wichtigsten sind. Die Datei wurde in ein XML-File umgewandelt um eine
reibungslose Weiterverarbeitung zu gewährleisten. Außerdem wurde die
Datei, durch entfernen von unnötigen Leerzeichen etc., etwas
bereinigt, um Speicherplatz einzusparen und eine höhere Effizienz zu
gewährleisten. Das Ergebnis ist in \autoref{fig:paperrank}
dargestellt.



Zum Abschluss wurden die entsprechenden gezählten Zitierungen in die
Ausgangsdateien geschrieben. Sie wurden in <entry>- und <counter>-Tags
verpackt, damit im Scoring Schritt ein reibungsloser Zugriff möglich
wird.  Nun soll noch ein kurzer Ausblick gegeben werden, was nicht in
das Resultat des Paperranks eingeflossen ist, aber die Ergebnisse noch
verbessern könnte. Bei der Extraktion der Quellen ist sicher noch
einiges an Optimierungspotenzial. Es könnten weitere Kriterien in die
Auswahl einfließen, um zum einen mehr Quellen zu finden und zum
anderen \emph{Nichtquellen} zu eliminieren. Außerdem wurde nicht
beachtet, welche Quelle die Zitierungen vornimmt. Generell hat sich
das Vorgehen an keinem Algorithmus, wie bspw. dem Random Surfer Modell
orientiert. Dies wäre auch ein interessanter Ansatz für weitere
Nachforschungen.  Nichtsdestotrotz entstand mit dieser Arbeit eine
weitere Möglichkeit die gegebenen wissenschaftlichen Texte zu bewerten
bzw. zu \emph{ranken}. Des Weiteren ist nun eine Aussage darüber
möglich, wie oft einzelne Paper in anderen Papern als Quellen
herangezogen werden und wie die Verteilung über alle Texte ist.



%%% Local Variables:
%%% mode: latex
%%% TeX-master: "../arbeit"
%%% End:


\chapter{Indexierung}\label{ch:indexing}
Die aus der Vorverarbeitung erstellten XML-Dokumente wurden mit \lucene \texttt{7.1.0} indiziert. 
Da einzelne Abschnitte bzw. Elemente des XML-Dokuments abgreifbar sind, können diese in einzelne Felder des Lucene-Dokuments gespeichert werden. 
Zu den Feldern, die aufgenommen wurden, gehören Titel, Author, Publikationsdatum und Content (Fließtext) einer Publikation, sowie Dateiname und -pfad des PDF-Dokuments, und die ID des XML-Dokuments. 
Des Weiteren wurde die Anzahl, wie oft eine Publikation von den anderen Publikationen in der PDF-Kollektion zitiert wurde, in ein Feld aufgenommen, um diesen Wert später beim Scoring nutzen zu können. 
Für die Indizierung wird die Klasse \texttt{de.uni\_leipzig.digital\_text\_forensics.lucene.Main} aufgerufen, welche wiederum die Klassen \texttt{de.uni\_leipzig.digital\_text\_forensics.lucene.XMLFileIndexer} aufruft, die als Konstruktor den Pfad zum Lucene-Index erhält. 

Die Suche ist in der Klasse \texttt{de.uni\_leipzig.digital\_text\_forensics.lucene.Searcher} implementiert. 
Um sowohl im Titel-Feld des Dokuments, als auch im Content-Feld suchen zu können, wurde das \texttt{MultiFieldQueryParser}-Objekt von \lucene genutzt. 
Um Suchanfragen, die zu Beginn des Dokuments vorkommen, höher zu gewichten, wurde das \texttt{SpanFirstQuery}-Objekt genutzt. 
Weiterhin werden Wörter, die im Titel-Feld des Dokuments stärker gewichtet als Wörter, die im Content-Feld des Dokuments vorkommen. 
Dazu wird dem Konstruktor des \texttt{MultiFieldQueryParser}-Objekts eine \texttt{HashMap} mit Key-Value-Paaren übergeben. 
Key ist das Feld (z.\,B. Titel), Value ist der Faktor mit dem das Feld gewichtet wird. 
Für das Titel-Feld wurde der Wert $0.8$ gewählt, für das Content-Feld $0.2$. 
Dies ist damit zu begründen, dass das Vorkommen der Suchanfrage, oder Teile davon, im Titel eines Dokuments ein Indikator für eine höhere Relevanz des Dokuments ist.  
Das Scoring für die Relevanz eines Dokuments für eine Suchanfrage setzt sich nun folgendermaßen zusammen: 
\begin{itemize} 
	\item Aus dem von \lucene intern berechneten Gewichtswert für die Relevanz eines Dokuments. 
	\item Aus der Anzahl, wie oft eine Publikation zitiert wurde. 
	\item Aus der Anzahl der Clicks, ein Dokument für eine Suchanfrage erhalten hat. 
	\item Aus der Zeitspanne, die Nutzer im Durchschnitt auf einem Dokument verbracht haben. 
\end{itemize}
Die letzten drei Faktoren der Liste werden dem von \lucene berechneten Gewichtswert aufaddiert. 
Werte für Clickzahl und Zeitspanne werden dabei aus dem Backend zur Verfügung gestellt. 
Die Anzahl, wie oft eine Publikation zitiert wurde, wird in der Vorverarbeitung direkt im XML-Dokument gespeichert und dadurch abgreifbar. 
Da die einzelnen Gewichtsfaktoren unterschiedliche Größenordnungen annehmen können, mussten sie normiert werden. 
Dazu wurde der Tangens hyperbolicus herangezogen, um Werte zwischen $0$ und $1$ zu erhalten. 
Dann wurde ein Gewichtswert festgelegt und aufmultipliziert. 

Für die Präsentation der Suchergebnisse werden Snippets generiert, die das Vorkommen der Suchanfrage in dem Dokument durch Highlighting der entsprechenden Wörter kenntlich macht. Dabei wurde die Länge eines Snippets auf $400$ Zeichen festgelegt. 
%%% Local Variables:
%%% mode: latex
%%% TeX-master: "../arbeit"
%%% End:


\chapter{Backend}\label{ch:backend}

%%% Local Variables:
%%% mode: latex
%%% TeX-master: "../arbeit"
%%% End:

\section{Auswahl eines Backend-Frameworks}
Zuerst wurde für die Programmierung einer Webanwendung ein geeignetes Framework gesucht, um Programmier-Paradigmen umzusetzen und die Architektur besser zu abstrahieren. 
Hierfür wurden zahlreiche Frameworks untersucht, welche Dependency Injection (DI), Inversion of Control (IoC) und Aspect-Oriented Programming (AOP) unterstützen.
Aufgrund der Auswahl der Programmiersprache Java für die Umsetzung der Anwendung schränkte sich die Anzahl der Frameworks ein.
Die Recherche ergab folgende drei Frameworks:

\begin{table}[h!]
	\centering
	
	\begin{tabularx}{\textwidth}{|l|c|X|}
		
		\hline
		\multicolumn{1}{|c|}{{\textbf{Frameworks}}} & \multicolumn{1}{c|}{{\textbf{Sprache}}} & \multicolumn{1}{c|}{{\textbf{Eigenschaft}}} \\
		\hline	     
		
		HiveMid & Java & DI, AOP-ähnliches Feature, IoC Container \\
		
		Google Guice & Java & DI, AOP, IoC Container,  Annotations, Generics, modular\\
		
		Spring Boot & Java & DI, AOP, IoC Container,  Annotations, Generics, modular\\
		\hline
	\end{tabularx}
	\caption{DI-Frameworks}
	\label{tbl:diFrameworks}
\end{table}

HiveMind und Google Guice bieten gegenüber Spring Boot leichter verständliche Programmierungstechniken sowie einen prägnanteren und lesbareren Code.
HiveMind fokussiert sich auf das Verbinden von Services. 
Seine Konfiguration erfolgt über eine XML-Datei oder eine eigene Definitions-Sprache. 
Hierdurch ist HiveMind ein kleiner und simpel gestalteter DI-Container.
Darüber hinaus  bietet HiveMind die Möglichkeit, mit AOP zu arbeiten.
Google Guice hingegen unterstützt Features wie Annotations und Generics, die ab Java 1.5 zur Verfügung stehen. 
Sie helfen dabei, eine weitgehend aufgeräumte und einfache Konfiguration zu ermöglichen.
Google Guice und Spring Boot bieten sehr ähnliche Ansätze und kommen mit vielen Anforderungen, die Unternehmenssoftware erfüllen müssen, zurecht.
Google Guice ist durch die geringere Komplexität leichter zu verstehen und insgesamt kleiner als Spring Boot.
Jedoch bietet die Modularität von Spring Boot den größeren Vorteil:
Die Module können, je nachdem welche der Entwickler benötigt, ohne viel Aufwand hinzugefügt werden.
Schlussendlich fiel die Entscheidung auf das Spring Boot-Framework, da dessen Flexibilität und Modularität die Entwicklung von Anwendungen stark vereinfacht und daher die beste Wahl darstellt.

\section{Auswahl einer Datenbank}
Als nächstes wurde für das spätere Speichern der Interaktion zwischen Backend, Suchergebnissen und User, genauer des User-Feedbacks eine geeignetes eingebettetes Datenbanksystem gesucht. 
Die Recherche ergab folgende Datenbanken:

\begin{table}[h!]
	\centering
	
	\begin{tabularx}{\textwidth}{|l|c|X|}
		
		\hline
		\multicolumn{1}{|c|}{{\textbf{Datenbank}}} & \multicolumn{1}{c|}{{\textbf{Sprache}}} & \multicolumn{1}{c|}{{\textbf{Eigenschaft}}} \\
		\hline	     
		SQLite & C & SQL-92-Standard,  Transaktionen, Unterabfragen (Subselects), Sichten (Views), Trigger und benutzerdefinierte Funktionen, direkte Integration in Anwendungen, In-Memory-Datenbank \\
		\hline
		Apache Cassandra & Java & Spaltenorientierte NoSQL-Datenbank, für sehr große strukturierte Datenbanken, hohe Skalierbarkeit und Ausfallsicherheit bei großen, verteilten Systemen \\
		\hline
		H2 & Java &   Schnell, Referenzielle Integrität, Transaktionen, Clustering, Datenkompression, Verschlüsselung und SSL, direkte Einbettung in Java-Anwendungen oder Betrieb als Server möglich, direkte Unterstützung in Spring Boot, In-Memory-Datenbank \\
		\hline
		
	\end{tabularx}
	\caption{Datenbanken}
	\label{tbl:dbs}
\end{table}

SQLite bietet einen leichten Einstieg in die Datenbanken.
Dabei stellt SQLite den größten Teil des SQL-92-Standards zur Verfügung und kann Transaktionen, Unterabfragen und viele weitere Funktionen durchführen.
Außerdem ist es eine In-Memory-Datenbank. 
Jedoch unterstützt Spring Boot diese Datenbank nicht von Haus aus und es müssten aufwendige Konfiguration vorgenommen werden.

Apache Cassandra ist eine spaltenorientierte NoSQL-Datenbank und ist für große strukturierte Daten, hohe Skalierbarkeit und Ausfallsicherheit ausgelegt.
Für das vorliegende Projekt ist Apache Cassandra jedoch zu groß ausgelegt, da für das Backend mit geringeren Datenmengen gearbeitet werden soll.

H2 ist eine In-Memory-Datenbank, welche schnell ist und referenzielle Integrität, Transaktionen, Clustering sowie  Datenkompression unterstützt.
Außerdem kann Spring Boot mit dieser Datenbank ohne besondere Maßnahmen wie aufwändige Konfigurationen verwendet und in die vorliegende Anwendung integriert werden.
Deshalb wurde entschlossen, H2 als Datenbank anzuwenden.

\section{Erstellung des Backends}
Nach der Auswahl der Backendtechnologien wurde die Grundarchitektur des Backends konzipiert und implementiert.

\subsection{Kommunikation mit der Datenbank}
Zunächst wurden Datenmodels wie \texttt{Query} oder \texttt{LoggingDocument} erstellt. 
Hieraus werden später die Tabellen der Datenbank generiert.
Um mit der Datenbank kommunizieren zu können, werden Data Access Objects (DAO) als Kommunikationsschnittstellen erstellt. 
Ein DAO hat eine Anbindung zu den Spring-Boot Repositorys, welche in der Lage sind SQL-Query zu generieren und übermittelt diese an die Datenbank.
Beispiele hierfür sind das Speichern und Abrufen von \texttt{LoggingDocument}-Daten, welche einen Teil des User-Feedbacks darstellen.
\begin{lstlisting}
public interface LoggingDocDao extends JpaRepository<LoggingDocument, Long> {
LoggingDocument findByDocId(Long docId);
}
\end{lstlisting}

Im obigen Code-Ausschnitt wird mit Hilfe von Spring Boot die SQL-Qyery \texttt{findByDocId} aus dem \texttt{LoggingDocument} generiert.
Dies findet über den Namen eines Interfaces statt.
Die einzelnen Komponenten, welche implementiert wurden, kommunizieren nicht direkt über die DAOs mit der Datenbank, sondern über ein Interface.
Dadurch ist eine lose Kopplung zwischen den Komponenten, DAOs und der Datenbank möglich.
Damit ist die Datenbank ohne große Änderungen in den Implementierungen austauschbar. 
Folglich fehlen nur noch Änderungen in den Konfigurationen und eventuell in den DAOs.
 
\begin{lstlisting}
public class LoggingDocServiceImpl implements LoggingDocService {
	public LoggingDocument findbyId(Long id) {
	return loggingDocDao.findOne(id);
	}}
\end{lstlisting}
Im vorliegenden Listing ist \texttt{LoggingDocService} als Beispiel für einen Service dargestellt.
In der Implantation des Interfaces wird nun das DAO aufgerufen, beispielsweise die Methode \texttt{findbyId}.

\subsection{Controller}
Im nächsten Schritt wurden sogenannte Controller erstellt.
Diese bilden eine wichtige Schnittstelle für die Kommunikation mit dem Frontend und Backend.
Controller reagieren auf HTTP-Requests, welche von dem Frontend oder anderen Clients gesendet werden.
Die Aufgabe ist es, für bestimmte Ressource-URLs spezielle Ereignisse auszuführen.
Ein Beispiel hierfür ist Auswertung der Suchanfrage der Search-Zeile im Frontend  und das Rücksenden der Suchergebnisse.
\begin{lstlisting}
@RequestMapping(method = RequestMethod.GET, path = "/")
public ModelAndView searchPage(
@RequestParam(defaultValue = "")
String query) {
ModelAndView modelAndView = new ModelAndView("search");
...
List<ScoreDoc> list = querySearcher.search(query);
...
modelAndView.addObject("searchResultPage", searchResultPage);
return modelAndView;}
\end{lstlisting}

Im obigen Code-Beispielabschnitt ist erkennbar, dass, beim Auslösen eines Request bei der Path-URL \texttt{\glqq/\grqq~} die Funktion \texttt{SearchPage} aufgerufen und eine Suche ausgeführt wird.
Hierfür wird der Request-Parameter mit \texttt{query} ausgewertet.
Die Suche erfolgt mithilfe der Komponente Lucene, welche bereits in Abschnitt \ref{ch:indexing} näher erläutert wurde.
Im Anschluss werden die Suchergebnisse als \texttt{modelAndView}-Objekt dem Frontend übergeben. 


\chapter{Evaluation und Fazit}\label{ch:results}
\section{Evaluation}


Nachdem die Suchmaschine komplett funktionsfähig war, wurde eine
Evaluation durchgeführt. Dies geschah um eine Aussage darüber treffen
zu können, wie gut die zurückgegebenen Ergebnisse sind.
Dazu wurden $40$ Topics erarbeitet und in die Suchmaschine
eingegeben. Diese bestanden auf einer Topic Nummer, erwarteten
Resultat und der Query, welches in die Suchleiste eingegeben
wurde. Daraufhin wurden jeweils die ersten $10$ Suchergebnisse von Hand
im Hinblick auf ihre Relevanz für den Suchbegriff klassifiziert. Eine
Tabelle, die alle Topics und Resultate auflistet ist auf Github
zu finden\footnote{siehe Link zur Tabelle mit MAP-Auswertung auf
\href{https://github.com/43ndr1k/Digital-Text-Forensics/blob/develop/topics/mean_average_precision.ods}{Github}.}.

Bei diesem Vorgehen wurde eine Mean Average Precision von $0,723$
erreicht.  Es ist erkennbar geworden, dass kurze Suchbegriffe, wie
\emph{Universität Weimar} und \emph{Text Classification} sehr gute
Ergebnisse lieferten. Alle Top-Resultate waren für sie relevant, was
zu einer Precision@10 von $1$ führte. Auch längere Queries wie
\emph{New methods for detecting and eliminating network
steganography} erreichte immerhin eine Precision@10 von $0,772$.
Im Gegensatz dazu, erzielte jedoch die Query \emph{Retrieval Model
Vector Space Model} lediglich eine Precision@10 von $0,192$. Dies ist
wahrscheinlich darauf zurückzuführen, dass dieses Model in der
digitalen Text-Forensik kaum genutzt wird. Daher ist es denkbar, dass
hierfür dennoch alle relevanten Dokumente gefunden wurden.

Zusammenfassend hat die Evaluation gezeigt, dass die Suchmaschine
bereits gute Ergebnisse liefert. Daher ist davon auszugehen, dass sich
die für das Ranking verwendeten Parameter, gut zur Bewertung von
wissenschaftlichen Publikationen eignet.

\section{Fazit}

Das Ziel dieses Praktikums ist es gewesen eine Suchmaschine zu
erstellen und zu implementieren, durch welche es möglich wird nach
wissenschaftlichen Publikationen aus dem Bereich der digitalen
Text-Forensik zu suchen.

Die Texte wurden in eine einheitliche Form gebracht und
indexiert. Während der Vorverarbeitung wurde außerdem auf die
Extraktion der Titel der Publikationen ein besonderer Wert gelegt, da
diese aus Gründen der Usability im Suchergebnis angezeigt werden und 
auch für das Ranking nicht unerheblich sind. Weiterhin wurde zur
Feststellung der Relevanz der bekannte BM25 Algorithmus
verwendet. Des Weiteren wurden Log-Daten, die Verweildauer auf einem
Dokument und die Anzahl der Erwähnungen der Paper in den anderen
wissenschaftlichen Texten verwendet.

Alle Daten werden in einer Datenbank gespeichert. Die Suchmaschine ist
voll funktionsfähig auf einem Webserver implementiert und kann über
eine grafische Weboberfläche genutzt werden. Neben dem Titel der zur
Suche relevanten Publikationen werden auch Snippets generiert, um eine
angenehmere und effizientere Nutzung zu gewährleisten.

In der Evaluation der Suchmaschine hat sich gezeigt, dass gute
Ergebnisse erzielt werden können, denn es befinden sich für nahezu
alle Suchanfragen relevante Ergebnisse unter den ersten
Suchergebnissen.  Wie bei fast allen Projekten in der Informatik
handelt es sich auch bei dieser Suchmaschine mehr um ein fortlaufendes
Projekt. Es können auf Grund von Nutzerfeedback immer weiter
Verbesserungen vorgenommen werden. Zurzeit wird daran gearbeitet, den
Upload von Texten durch den Nutzer zu gewährleisten und diese mit in
die Suchmaschine einfließen zu lassen. 
Außerdem kann es in der zukünftigen Bearbeitung eine sinnvolle
Erweiterung sein, nach weiteren Feldern suchen zu können, wie bspw.
Konferenz.

Zusammenfassend kann gesagt werden, dass es unserem Team im Laufe des
Praktikums gelungen ist, eine Suchmaschine zu entwickeln, welche die
an sie gestellten Anforderungen voll und ganz erfüllt. Es wird durch
sie möglich auf einer großen Menge von wissenschaftlichen Texten eine
Suche durchzuführen und relevante Ergebnisse zu erhalten.

%%% Local Variables:
%%% mode: latex
%%% TeX-master: "../arbeit"
%%% End:




%%% könnt ihr gerne benutzen, seh hier aber keinen Mehrwert.

% Literaturverzeichnis
%\bibliography{bibliography}
% \printbibliography[heading=bibintoc]

\newpage
\appendix



\end{document}

%%% Local Variables:
%%% mode: latex
%%% TeX-master: t
%%% End:
