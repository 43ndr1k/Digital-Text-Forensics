Die aus der Vorverarbeitung erstellten XML-Dokumente werden mit
\lucene \lstinline{7.1.0} indiziert.  Da einzelne Abschnitte
bzw. Elemente des XML-Dokuments abgreifbar sind, können sie in
einzelne Felder des Lucene-Dokuments gespeichert werden.  Zu den
Feldern, die aufgenommen wurden gehören Titel, Autor,
Publikationsdatum und der Content als Fließtext einer Publikation,
sowie Dateiname und -pfad des PDF-Dokuments, und die ID des
XML-Dokuments.  Des Weiteren wurde die Anzahl, wie oft eine
Publikation von den anderen Publikationen in der PDF-Kollektion
zitiert wurde, in ein Feld aufgenommen, um diesen Wert später beim
Scoring nutzen zu können.  Für die Indizierung wird die Klasse
\lstinline{de.uni_leipzig.digital_text_forensics.lucene.Main}
aufgerufen, welche wiederum die Klassen
\lstinline{de.uni_leipzig.digital_text_forensics.lucene.XMLFileIndexer}
aufruft, die als Konstruktor den Pfad zum Lucene-Index erhält.

Die Suche ist in der Klasse \lstinline{de.uni_leipzig.digital_text_forensics.lucene.Searcher} implementiert. 
Um sowohl im Titel-Feld des Dokuments, als auch im Content-Feld suchen zu können, wurde das \lstinline{MultiFieldQueryParser}-Objekt von \lucene genutzt. 
Um Suchanfragen, die zu Beginn des Dokuments vorkommen, höher zu gewichten, wurde das \lstinline{SpanFirstQuery}-Objekt genutzt. 
Weiterhin werden Wörter, die im Titel-Feld des Dokuments stärker gewichtet als Wörter, die im Content-Feld des Dokuments vorkommen. 
Dazu wird dem Konstruktor des \lstinline{MultiFieldQueryParser}-Objekts eine \lstinline{HashMap} mit Key-Value-Paaren übergeben. 
Key ist das Feld (z.\,B. Titel), Value ist der Faktor mit dem das Feld gewichtet wird. 
Für das Titel-Feld wurde der Wert $0.8$ gewählt, für das Content-Feld $0.2$. 
Dies ist damit zu begründen, dass das Vorkommen der Suchanfrage, oder Teile davon, im Titel eines Dokuments ein Indikator für eine höhere Relevanz des Dokuments ist.  
Das Scoring für die Relevanz eines Dokuments für eine Suchanfrage setzt sich nun folgendermaßen zusammen: 
\begin{itemize} 
	\item Aus dem von \lucene intern berechneten Gewichtswert für die Relevanz eines Dokuments. 
	\item Aus der Anzahl, wie oft eine Publikation zitiert wurde. 
	\item Aus der Anzahl der Clicks, ein Dokument für eine Suchanfrage erhalten hat. 
	\item Aus der Zeitspanne, die Nutzer im Durchschnitt auf einem Dokument verbracht haben. 
\end{itemize}
Die letzten drei Faktoren der Liste werden dem von \lucene berechneten Gewichtswert aufaddiert. 
Werte für Clickzahl und Zeitspanne werden dabei aus dem Backend zur Verfügung gestellt. 
Die Anzahl, wie oft eine Publikation zitiert wurde, wird in der Vorverarbeitung direkt im XML-Dokument gespeichert und dadurch abgreifbar. 
Da die einzelnen Gewichtsfaktoren unterschiedliche Größenordnungen annehmen können, mussten sie normiert werden. 
Dazu wurde der Tangens hyperbolicus herangezogen, um Werte zwischen $0$ und $1$ zu erhalten. 
Dann wurde ein Gewichtswert festgelegt und aufmultipliziert. 

Für die Präsentation der Suchergebnisse werden Snippets generiert, die das Vorkommen der Suchanfrage in dem Dokument durch Highlighting der entsprechenden Wörter kenntlich macht. Dabei wurde die Länge eines Snippets auf $400$ Zeichen festgelegt. 
%%% Local Variables:
%%% mode: latex
%%% TeX-master: "../arbeit"
%%% End:
