
\section{Fazit}

Das Ziel dieses Praktikums ist es gewesen eine Suchmaschine zu
erstellen und zu implementieren, durch welche es möglich wird nach
wissenschaftlichen Publikationen aus dem Bereich der digitalen
Text-Forensik zu suchen.

Die Texte wurden in eine einheitliche Form gebracht und
indexiert. Während der Vorverarbeitung wurde außerdem auf die
Extraktion der Titel der Publikationen ein besonderer Wert gelegt, da
diese aus Gründen der Usability im Suchergebnis angezeigt werden und 
auch für das Ranking nicht unerheblich sind. Weiterhin wurde zur
Feststellung der Relevanz der bekannte BM25 Algorithmus
verwendet. Des Weiteren wurden Log-Daten, die Verweildauer auf einem
Dokument und die Anzahl der Erwähnungen der Paper in den anderen
wissenschaftlichen Texten verwendet.

Alle Daten werden in einer Datenbank gespeichert. Die Suchmaschine ist
voll funktionsfähig auf einem Webserver implementiert und kann über
eine grafische Weboberfläche genutzt werden. Neben dem Titel der zur
Suche relevanten Publikationen werden auch Snippets generiert, um eine
angenehmere und effizientere Nutzung zu gewährleisten.

In der Evaluation der Suchmaschine hat sich gezeigt, dass gute
Ergebnisse erzielt werden können, denn es befinden sich für nahezu
alle Suchanfragen relevante Ergebnisse unter den ersten
Suchergebnissen.  Wie bei fast allen Projekten in der Informatik
handelt es sich auch bei dieser Suchmaschine mehr um ein fortlaufendes
Projekt. Es können auf Grund von Nutzerfeedback immer weiter
Verbesserungen vorgenommen werden. Zurzeit wird daran gearbeitet, den
Upload von Texten durch den Nutzer zu gewährleisten und diese mit in
die Suchmaschine einfließen zu lassen. 
Außerdem kann es in der zukünftigen Bearbeitung eine sinnvolle
Erweiterung sein, nach weiteren Feldern suchen zu können, wie bspw.
Konferenz.

Zusammenfassend kann gesagt werden, dass es unserem Team im Laufe des
Praktikums gelungen ist, eine Suchmaschine zu entwickeln, welche die
an sie gestellten Anforderungen voll und ganz erfüllt. Es wird durch
sie möglich auf einer großen Menge von wissenschaftlichen Texten eine
Suche durchzuführen und relevante Ergebnisse zu erhalten.

%%% Local Variables:
%%% mode: latex
%%% TeX-master: "../arbeit"
%%% End:
