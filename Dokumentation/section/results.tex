\section{Evaluation}


Nachdem die Suchmaschine komplett funktionsfähig war, wurde eine
Evaluation durchgeführt. Dies geschah um eine Aussage darüber treffen
zu können, wie gut die zurückgegebenen Ergebnisse sind.
Dazu wurden $40$ Topics erarbeitet und in die Suchmaschine
eingegeben. Diese bestanden auf einer Topic Nummer, erwarteten
Resultat und der Query, welches in die Suchleiste eingegeben
wurde. Daraufhin wurden jeweils die ersten $10$ Suchergebnisse von Hand
im Hinblick auf ihre Relevanz für den Suchbegriff klassifiziert. Eine
Tabelle, die alle Topics und Resultate auflistet ist auf Github
zu finden\footnote{siehe Link zur Tabelle mit MAP-Auswertung auf
\href{https://github.com/43ndr1k/Digital-Text-Forensics/blob/develop/topics/mean_average_precision.ods}{Github}.}.

Bei diesem Vorgehen wurde eine Mean Average Precision von $0,723$
erreicht.  Es ist erkennbar geworden, dass kurze Suchbegriffe, wie
\emph{Universität Weimar} und \emph{Text Classification} sehr gute
Ergebnisse lieferten. Alle Top-Resultate waren für sie relevant, was
zu einer Precision@10 von $1$ führte. Auch längere Queries wie
\emph{New methods for detecting and eliminating network
steganography} erreichte immerhin eine Precision@10 von $0,772$.
Im Gegensatz dazu, erzielte jedoch die Query \emph{Retrieval Model
Vector Space Model} lediglich eine Precision@10 von $0,192$. Dies ist
wahrscheinlich darauf zurückzuführen, dass dieses Model in der
digitalen Text-Forensik kaum genutzt wird. Daher ist es denkbar, dass
hierfür dennoch alle relevanten Dokumente gefunden wurden.

Zusammenfassend hat die Evaluation gezeigt, dass die Suchmaschine
bereits gute Ergebnisse liefert. Daher ist davon auszugehen, dass sich
die für das Ranking verwendeten Parameter, gut zur Bewertung von
wissenschaftlichen Publikationen eignet.

\section{Fazit}

Das Ziel dieses Praktikums ist es gewesen eine Suchmaschine zu
erstellen und zu implementieren, durch welche es möglich wird nach
wissenschaftlichen Publikationen aus dem Bereich der digitalen
Text-Forensik zu suchen.

Die Texte wurden in eine einheitliche Form gebracht und
indexiert. Während der Vorverarbeitung wurde außerdem auf die
Extraktion der Titel der Publikationen ein besonderer Wert gelegt, da
diese aus Gründen der Usability im Suchergebnis angezeigt werden und 
auch für das Ranking nicht unerheblich sind. Weiterhin wurde zur
Feststellung der Relevanz der bekannte BM25 Algorithmus
verwendet. Des Weiteren wurden Log-Daten, die Verweildauer auf einem
Dokument und die Anzahl der Erwähnungen der Paper in den anderen
wissenschaftlichen Texten verwendet.

Alle Daten werden in einer Datenbank gespeichert. Die Suchmaschine ist
voll funktionsfähig auf einem Webserver implementiert und kann über
eine grafische Weboberfläche genutzt werden. Neben dem Titel der zur
Suche relevanten Publikationen werden auch Snippets generiert, um eine
angenehmere und effizientere Nutzung zu gewährleisten.

In der Evaluation der Suchmaschine hat sich gezeigt, dass gute
Ergebnisse erzielt werden können, denn es befinden sich für nahezu
alle Suchanfragen relevante Ergebnisse unter den ersten
Suchergebnissen.  Wie bei fast allen Projekten in der Informatik
handelt es sich auch bei dieser Suchmaschine mehr um ein fortlaufendes
Projekt. Es können auf Grund von Nutzerfeedback immer weiter
Verbesserungen vorgenommen werden. Zurzeit wird daran gearbeitet, den
Upload von Texten durch den Nutzer zu gewährleisten und diese mit in
die Suchmaschine einfließen zu lassen. 
Außerdem kann es in der zukünftigen Bearbeitung eine sinnvolle
Erweiterung sein, nach weiteren Feldern suchen zu können, wie bspw.
Konferenz.

Zusammenfassend kann gesagt werden, dass es unserem Team im Laufe des
Praktikums gelungen ist, eine Suchmaschine zu entwickeln, welche die
an sie gestellten Anforderungen voll und ganz erfüllt. Es wird durch
sie möglich auf einer großen Menge von wissenschaftlichen Texten eine
Suche durchzuführen und relevante Ergebnisse zu erhalten.

%%% Local Variables:
%%% mode: latex
%%% TeX-master: "../arbeit"
%%% End:
