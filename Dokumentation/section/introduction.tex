Ziel des Praktikums war es, eine themenspezifische Suchmaschine zu
implementieren, die wissenschaftliche Publikationen aus dem Fachgebiet
der digitalen Text-Forensik indexieren und durchsuchen kann. In der
digitalen Text-Forensik werden Methoden und Verfahren zur
Identifikation von Autoren, Erkennung von Plagiaten und
Profilerstellung von Autoren untersucht und bereitgestellt. Die
Publikationen lagen hauptsächlich im PDF-Format und in englischer
Sprache vor, ein geringer Anteil in weiteren Sprachen wie deutsch, französisch und griechisch, sodass der Fokus auf diesen Sprachen liegt. Dokumente, deren Kodierung nicht erkennbar ist, werden nicht
berücksichtigt. 

Im Verlauf der Vorverarbeitung wurden die Dokumente in ein
einheitliches Format und eine einheitliche Zeichenkodierung überführt.
Um einzelne Bestandteile bzw. Felder eines Dokuments, wie z.\,B. Titel
oder Fließtext, getrennt betrachten zu können, wurde XML als Format
gewählt, in das alle Dokumente konvertiert werden und dessen Instanzen
indiziert werden.  Die Erkennung und Extraktion einzelner Bestandteile eines
Dokuments war dabei eine wesentliche Aufgabe, da Titel der
Publikationen als Suchresultat angezeigt werden sollen, diese aber
kaum als Meta-Informationen in den PDF-Dokumenten vorhanden waren.
Das Vorkommen von Wörtern der Suchanfrage in bestimmten Feldern kann
somit beim Ranking berücksichtigt werden: Wenn Wörter der Suchanfrage
im Titel eines Dokuments vorkommen, wird es als relevanter bewertet,
als wenn sie im Fließtext vorkommen.

Weitere Parameter, die in die Gewichtung der Relevanz eines Dokuments
eingehen sind Logdaten, z.\,B. wie oft ein Dokument für eine Suchanfrage
geklickt wurde, sowie die Zeitdauer, die Nutzer auf einem Dokument
verbracht haben.  Die Aufzeichnung dieser Daten wurden in einer
Datenbank im Backend implementiert, die auch genutzt wurde, um die
Autovervollständigung von Suchanfragen zu ermöglichen.  Weiterhin geht
in die Gewichtung eines Dokuments die Anzahl der Erwähnungen des
Dokuments in anderen Publikationen mit ein.  Zur Ermittlung dieses
Kennwertes wurde ein Perl-Skript geschrieben, sodass dieser Wert bei
Erweiterung des Indexes durch neue Dokumente, aktualisiert werden
kann.

Die Oben genannte Komponenten der Suchmaschine wurden unter Verwendung
verschiedener Programm-Bilbiotheken implementiert, darunter \pdfbox,
\tika und Docear bei der Vorverarbeitung, \lucene für
Indexierung und Suche und das Spring Framework für Backend und
Frontend.  Das Frontend dient als Interface für Suchanfragen eines
Nutzers und zur Präsentation der Suchergebnisse, geordnet nach
Relevanz und mit einem kurzen Textausschnitt (Snippet) für jedes
Suchergebnis.  In den folgenden Kapiteln wird auf die einzelnen
Komponenten der Suchmaschine detailliert eingegangen.

%%% Local Variables:
%%% mode: latex
%%% TeX-master: "../arbeit"
%%% End:
