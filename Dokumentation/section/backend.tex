
%%% Local Variables:
%%% mode: latex
%%% TeX-master: "../arbeit"
%%% End:

\section{Auswahl eines Backend-Frameworks}
Zuerst wurde für die Programmierung einer Webanwendung ein geeignetes Framework gesucht, um Programmier-Paradigmen umzusetzen und die Architektur besser zu abstrahieren. 
Hierfür wurden zahlreiche Frameworks untersucht, welche Dependency Injection (DI), Inversion of Control (IoC) und Aspect-Oriented Programming (AOP) unterstützen.
Aufgrund der Auswahl der Programmiersprache Java für die Umsetzung der Anwendung schränkte sich die Anzahl der Frameworks ein.
Die Recherche ergab folgende drei Frameworks:

\begin{table}[h!]
	\centering
	
	\begin{tabularx}{\textwidth}{|l|c|X|}
		
		\hline
		\multicolumn{1}{|c|}{{\textbf{Frameworks}}} & \multicolumn{1}{c|}{{\textbf{Sprache}}} & \multicolumn{1}{c|}{{\textbf{Eigenschaft}}} \\
		\hline	     
		
		HiveMid & Java & DI, AOP-ähnliches Feature, IoC Container \\
		
		Google Guice & Java & DI, AOP, IoC Container,  Annotations, Generics, modular\\
		
		Spring Boot & Java & DI, AOP, IoC Container,  Annotations, Generics, modular\\
		\hline
	\end{tabularx}
	\caption{DI-Frameworks}
	\label{tbl:diFrameworks}
\end{table}

HiveMind und Google Guice bieten gegenüber Spring Boot leichter verständliche Programmierungstechniken sowie einen prägnanteren und lesbareren Code.
HiveMind fokussiert sich auf das Verbinden von Services. 
Seine Konfiguration erfolgt über eine XML-Datei oder eine eigene Definitions-Sprache. 
Hierdurch ist HiveMind ein kleiner und simpel gestalteter DI-Container.
Darüber hinaus  bietet HiveMind die Möglichkeit, mit AOP zu arbeiten.
Google Guice hingegen unterstützt Features wie Annotations und Generics, die ab Java 1.5 zur Verfügung stehen. 
Sie helfen dabei, eine weitgehend aufgeräumte und einfache Konfiguration zu ermöglichen.
Google Guice und Spring Boot bieten sehr ähnliche Ansätze und kommen mit vielen Anforderungen, die Unternehmenssoftware erfüllen müssen, zurecht.
Google Guice ist durch die geringere Komplexität leichter zu verstehen und insgesamt kleiner als Spring Boot.
Jedoch bietet die Modularität von Spring Boot den größeren Vorteil:
Die Module können, je nachdem welche der Entwickler benötigt, ohne viel Aufwand hinzugefügt werden.
Schlussendlich fiel die Entscheidung auf das Spring Boot-Framework, da dessen Flexibilität und Modularität die Entwicklung von Anwendungen stark vereinfacht und daher die beste Wahl darstellt.

\section{Auswahl einer Datenbank}
Als nächstes wurde für das spätere Speichern der Interaktion zwischen Backend, Suchergebnissen und User, genauer des User-Feedbacks eine geeignetes eingebettetes Datenbanksystem gesucht. 
Die Recherche ergab folgende Datenbanken:

\begin{table}[h!]
	\centering
	
	\begin{tabularx}{\textwidth}{|l|c|X|}
		
		\hline
		\multicolumn{1}{|c|}{{\textbf{Datenbank}}} & \multicolumn{1}{c|}{{\textbf{Sprache}}} & \multicolumn{1}{c|}{{\textbf{Eigenschaft}}} \\
		\hline	     
		SQLite & C & SQL-92-Standard,  Transaktionen, Unterabfragen (Subselects), Sichten (Views), Trigger und benutzerdefinierte Funktionen, direkte Integration in Anwendungen, In-Memory-Datenbank \\
		\hline
		Apache Cassandra & Java & Spaltenorientierte NoSQL-Datenbank, für sehr große strukturierte Datenbanken, hohe Skalierbarkeit und Ausfallsicherheit bei großen, verteilten Systemen \\
		\hline
		H2 & Java &   Schnell, Referenzielle Integrität, Transaktionen, Clustering, Datenkompression, Verschlüsselung und SSL, direkte Einbettung in Java-Anwendungen oder Betrieb als Server möglich, direkte Unterstützung in Spring Boot, In-Memory-Datenbank \\
		\hline
		
	\end{tabularx}
	\caption{Datenbanken}
	\label{tbl:dbs}
\end{table}

SQLite bietet einen leichten Einstieg in die Datenbanken.
Dabei stellt SQLite den größten Teil des SQL-92-Standards zur Verfügung und kann Transaktionen, Unterabfragen und viele weitere Funktionen durchführen.
Außerdem ist es eine In-Memory-Datenbank. 
Jedoch unterstützt Spring Boot diese Datenbank nicht von Haus aus und es müssten aufwendige Konfiguration vorgenommen werden.

Apache Cassandra ist eine spaltenorientierte NoSQL-Datenbank und ist für große strukturierte Daten, hohe Skalierbarkeit und Ausfallsicherheit ausgelegt.
Für das vorliegende Projekt ist Apache Cassandra jedoch zu groß ausgelegt, da für das Backend mit geringeren Datenmengen gearbeitet werden soll.

H2 ist eine In-Memory-Datenbank, welche schnell ist und referenzielle Integrität, Transaktionen, Clustering sowie  Datenkompression unterstützt.
Außerdem kann Spring Boot mit dieser Datenbank ohne besondere Maßnahmen wie aufwändige Konfigurationen verwendet und in die vorliegende Anwendung integriert werden.
Deshalb wurde entschlossen, H2 als Datenbank anzuwenden.

\section{Erstellung des Backends}
Nach der Auswahl der Backendtechnologien wurde die Grundarchitektur des Backends konzipiert und implementiert.

\subsection{Kommunikation mit der Datenbank}
Zunächst wurden Datenmodels wie \texttt{Query} oder \texttt{LoggingDocument} erstellt. 
Hieraus werden später die Tabellen der Datenbank generiert.
Um mit der Datenbank kommunizieren zu können, werden Data Access Objects (DAO) als Kommunikationsschnittstellen erstellt. 
Ein DAO hat eine Anbindung zu den Spring-Boot Repositorys, welche in der Lage sind SQL-Query zu generieren und übermittelt diese an die Datenbank.
Beispiele hierfür sind das Speichern und Abrufen von \texttt{LoggingDocument}-Daten, welche einen Teil des User-Feedbacks darstellen.
\begin{lstlisting}
public interface LoggingDocDao extends JpaRepository<LoggingDocument, Long> {
LoggingDocument findByDocId(Long docId);
}
\end{lstlisting}

Im obigen Code-Ausschnitt wird mit Hilfe von Spring Boot die SQL-Qyery \texttt{findByDocId} aus dem \texttt{LoggingDocument} generiert.
Dies findet über den Namen eines Interfaces statt.
Die einzelnen Komponenten, welche implementiert wurden, kommunizieren nicht direkt über die DAOs mit der Datenbank, sondern über ein Interface.
Dadurch ist eine lose Kopplung zwischen den Komponenten, DAOs und der Datenbank möglich.
Damit ist die Datenbank ohne große Änderungen in den Implementierungen austauschbar. 
Folglich fehlen nur noch Änderungen in den Konfigurationen und eventuell in den DAOs.
 
\begin{lstlisting}
public class LoggingDocServiceImpl implements LoggingDocService {
	public LoggingDocument findbyId(Long id) {
	return loggingDocDao.findOne(id);
	}}
\end{lstlisting}
Im vorliegenden Listing ist \texttt{LoggingDocService} als Beispiel für einen Service dargestellt.
In der Implantation des Interfaces wird nun das DAO aufgerufen, beispielsweise die Methode \texttt{findbyId}.

\subsection{Controller}
Im nächsten Schritt wurden sogenannte Controller erstellt.
Diese bilden eine wichtige Schnittstelle für die Kommunikation mit dem Frontend und Backend.
Controller reagieren auf HTTP-Requests, welche von dem Frontend oder anderen Clients gesendet werden.
Die Aufgabe ist es, für bestimmte Ressource-URLs spezielle Ereignisse auszuführen.
Ein Beispiel hierfür ist Auswertung der Suchanfrage der Search-Zeile im Frontend  und das Rücksenden der Suchergebnisse.
\begin{lstlisting}
@RequestMapping(method = RequestMethod.GET, path = "/")
public ModelAndView searchPage(
@RequestParam(defaultValue = "")
String query) {
ModelAndView modelAndView = new ModelAndView("search");
...
List<ScoreDoc> list = querySearcher.search(query);
...
modelAndView.addObject("searchResultPage", searchResultPage);
return modelAndView;}
\end{lstlisting}

Im obigen Code-Beispielabschnitt ist erkennbar, dass, beim Auslösen eines Request bei der Path-URL \texttt{\glqq/\grqq~} die Funktion \texttt{SearchPage} aufgerufen und eine Suche ausgeführt wird.
Hierfür wird der Request-Parameter mit \texttt{query} ausgewertet.
Die Suche erfolgt mithilfe der Komponente Lucene, welche bereits in Abschnitt \ref{ch:indexing} näher erläutert wurde.
Im Anschluss werden die Suchergebnisse als \texttt{modelAndView}-Objekt dem Frontend übergeben. 
